\documentclass{article}
\usepackage[utf8]{inputenc}
\usepackage[brazil]{babel}
\usepackage{setspace}
\usepackage{mathtools}
\usepackage{pgfplots}
\usepackage{listings}
\usepackage{xcolor}
\usepackage{natbib}
\usepackage{graphicx}

\DeclarePairedDelimiter\ceil{\lceil}{\rceil}
\DeclarePairedDelimiter\floor{\lfloor}{\rfloor}

\definecolor{codegreen}{rgb}{0,0.6,0}
\definecolor{codegray}{rgb}{0.5,0.5,0.5}
\definecolor{codepurple}{rgb}{0.58,0,0.82}
\definecolor{backcolour}{rgb}{0.95,0.95,0.92}

\lstdefinestyle{codigo}{
    numberstyle=\tiny,
    basicstyle=\ttfamily\footnotesize,
    breakatwhitespace=false,         
    breaklines=true,                 
    captionpos=b,                    
    keepspaces=true,                 
    numbers=left,                    
    numbersep=5pt,                  
    showspaces=false,                
    showstringspaces=false,
    showtabs=false,                  
    tabsize=2
}

\lstset{style=codigo}

\setstretch{1.5}
\pgfplotsset{width=10cm,compat=1.9}

\pagenumbering{gobble}
\clearpage
\thispagestyle{empty}

\title{Qualidade de Software}
\author{
        Lucas Santiago de Oliveira\and
        Thiago Henriques Nogueira\and 
        Tallys Assis de Souza\and 
        Rafael Tristão Schettino
    }
        
\date{Março de 2020}



\begin{document}
\maketitle

%INTRODUÇÃO
\hspace{6pt} O artigo abordado nesse trabalho, consiste em um novo modelo para o desenvolvimento de softwares, denominado como \emph{AZ-Model}''. A ideia principal do projeto é substituir os métodos usados atualmente, que estão muito focados nos benefícios pessoais de cada empresa, para um modelo padrão mundial que é extremamente eficiente as diversas eventualidades.

%CONTEXTO & DESENVOLVIMENTO
\hspace{6pt} O modelo \emph{AZ-Model}'' consiste na participação e consentimento de todos os setores nas etapas de desenvolvimento dos softwares, fazendo assim o produto ficar mais interligado. A ideia inicial veio dos problemas de grandes empresas como a \emph{Microsoft} que por conta do seu número de funcionários dificulta o entendimento de todos na confecção do \emph{software}. Um problema citado recorrentemente foi o planejamento do produto estar em desacordo com a venda do mesmo, fazendo a empresa perder uma fração do lucro. 

%RESULTADOS
\hspace{6pt} Feita algumas pesquisas por amostragem, em um grande número de empresas, foi possível notar a eficácia do \emph{Modelo-AZ}. Entre os ganhos, a implementação mostrou-se superior durante todo o processo de desenvolvimento de um \emph{software} se comparado com o modelo atual.

%CONCLUSAO
\hspace{6pt} A metodologia utilizada é muito importante no desenvolvimento de \emph{softwares}. Sendo assim, o modelo em questão se mostrou eficiente, uma vez que consegue unir todos os setores e tem um engajamento do cliente mais perceptível perante o produto final. Dessa forma, o produto é mais focado no interesse do cliente, criando uma maior confiança desse no produto que será lançado.

\end{document}
