\documentclass{article}
\usepackage[utf8]{inputenc}
\usepackage[brazil]{babel}
\usepackage{setspace}
\usepackage{mathtools}
\usepackage{pgfplots}
\usepackage{listings}
\usepackage{xcolor}
\usepackage{natbib}
\usepackage{graphicx}

\DeclarePairedDelimiter\ceil{\lceil}{\rceil}
\DeclarePairedDelimiter\floor{\lfloor}{\rfloor}

\definecolor{codegreen}{rgb}{0,0.6,0}
\definecolor{codegray}{rgb}{0.5,0.5,0.5}
\definecolor{codepurple}{rgb}{0.58,0,0.82}
\definecolor{backcolour}{rgb}{0.95,0.95,0.92}

\lstdefinestyle{codigo}{
    backgroundcolor=\color{backcolour},   
    commentstyle=\color{codegreen},
    keywordstyle=\color{magenta},
    numberstyle=\tiny\color{codegray},
    stringstyle=\color{codepurple},
    basicstyle=\ttfamily\footnotesize,
    breakatwhitespace=false,         
    breaklines=true,                 
    captionpos=b,                    
    keepspaces=true,                 
    numbers=left,                    
    numbersep=5pt,                  
    showspaces=false,                
    showstringspaces=false,
    showtabs=false,                  
    tabsize=2
}

\lstset{style=codigo}

\setstretch{1.5}
\pgfplotsset{width=10cm,compat=1.9}

\title{Questões básicas de AEDs 2}
\author{Lucas Santiago de Oliveira}
\date{Março 2020}

\begin{document}

\maketitle
\tableofcontents
\newpage

\section{Questão 1}
    a) $2^0 = 1$ \newline
    b) $2^1 = 2$ \newline
    c) $2^2 = 4$ \newline
    d) $2^3 = 8$ \newline
    e) $2^4 = 16$ \newline
    f) $2^5 = 32$ \newline
    g) $2^6 = 64$ \newline
    h) $2^7 = 128$ \newline
    i) $2^8 = 256$ \newline
    j) $2^9 = 512$ \newline
    k) $2^10 = 1024$ \newline
    l) $2^11 = 2048$

\section{Questão 2}
    a) $\log_2 2048 = 11$ \newline
    b) $\log_2 1024 = 10$ \newline
    c) $\log_2 512 = 9$ \newline
    d) $\log_2 256 = 8$ \newline
    e) $\log_2 128 = 7$ \newline
    f) $\log_2 64 = 6$ \newline
    g) $\log_2 32 = 5$ \newline
    h) $\log_2 16 = 4$ \newline
    i) $\log_2 8 = 3$ \newline
    j) $\log_2 4 = 2$ \newline
    k) $\log_2 2 = 1$ \newline
    l) $\log_2 1 = 0$
    
\section{Questão 3}
    a) $\ceil{4,01} = 5$ \newline
    b) $\floor{4,01} = 4$ \newline
    c) $\ceil{4,99} = 5$ \newline
    d) $\floor{4,99} = 4$ \newline
    e) $\log_2 16 = 4$ \newline
    f) $\ceil{\log_2 16} = 4$ \newline
    g) $\floor{\log_2 16} = 4$ \newline
    h) $\log_2 17 = 4.08746284125$ \newline
    i) $\ceil{\log_2 17} = 5$ \newline
    j) $\floor{\log_2 17} = 4$ \newline
    k) $\log_2 15 = 3.90689059561$ \newline
    l) $\ceil{\log_2 15} = 4$ \newline
    m) $\floor{\log_2 15} = 3$
    
\section{Questão 4}
    a) 
\begin{tikzpicture}
\begin{axis}[xlabel=$x$,ylabel=$y$,
xmin=-10,xmax=10,ymin=-10,ymax=10, axis lines=center, axis equal]
\addplot[color=blue,smooth,ultra thick, domain=-10:10] {x};
\end{axis}
\end{tikzpicture}
\newline
    
    b) 
\begin{tikzpicture}
\begin{axis}[xlabel=$x$,ylabel=$y$,
xmin=-10,xmax=10,ymin=-10,ymax=10, axis lines=center, axis equal]
\addplot[color=blue,smooth,ultra thick, domain=-10:10] {x^2};
\end{axis}
\end{tikzpicture}
\newline
    
    c) 
\begin{tikzpicture}
\begin{axis}[xlabel=$x$,ylabel=$y$,
xmin=-10,xmax=10,ymin=-10,ymax=10, axis lines=center, axis equal]
\addplot[color=blue,smooth,ultra thick, domain=-10:10] {x^3};
\end{axis}
\end{tikzpicture}
\newline

    d) 
\begin{tikzpicture}
\begin{axis}[xlabel=$x$,ylabel=$y$,
xmin=-10,xmax=10,ymin=-10,ymax=10, axis lines=center, axis equal]
\addplot[color=blue,smooth,ultra thick, domain=-10:10] {sqrt(x)};
\end{axis}
\end{tikzpicture}
\newline

    e) 
\begin{tikzpicture}
\begin{axis}[xlabel=$x$,ylabel=$y$,
xmin=-10,xmax=10,ymin=-10,ymax=10, axis lines=center, axis equal]
\addplot[color=blue,smooth,ultra thick, domain=-10:10] {log2(x)};
\end{axis}
\end{tikzpicture}
\newline

    f) 
\begin{tikzpicture}
\begin{axis}[xlabel=$x$,ylabel=$y$,
xmin=-10,xmax=10,ymin=-10,ymax=10, axis lines=center, axis equal]
\addplot[color=blue,smooth,ultra thick, domain=-10:10] {3*x^2 + 5*x - 3};
\end{axis}
\end{tikzpicture}
\newline

    g)
\begin{tikzpicture}
\begin{axis}[xlabel=$x$,ylabel=$y$,
xmin=-10,xmax=10,ymin=-10,ymax=10, axis lines=center, axis equal]
\addplot[color=blue,smooth,ultra thick, domain=-10:10] {-3*x^2+5*x-3};
\end{axis}
\end{tikzpicture}
\newline

    h)
\begin{tikzpicture}
\begin{axis}[xlabel=$x$,ylabel=$y$,
xmin=-10,xmax=10,ymin=-10,ymax=10, axis lines=center, axis equal]
\addplot[color=blue,smooth,ultra thick, domain=0:10] {x^2};
\addplot[color=blue,smooth,ultra thick, domain=-10:0] {x^2};
\end{axis}
\end{tikzpicture}
\newline

    i)
\begin{tikzpicture}
\begin{axis}[xlabel=$x$,ylabel=$y$,
xmin=-10,xmax=10,ymin=-10,ymax=10, axis lines=center, axis equal]
\addplot[color=blue,smooth,ultra thick, domain=-10:10] {5*x^4+2*x^2};
\end{axis}
\end{tikzpicture}
\newline

    j)
\begin{tikzpicture}
\begin{axis}[xlabel=$x$,ylabel=$y$,
xmin=-10,xmax=10,ymin=-10,ymax=10, axis lines=center, axis equal]
\addplot[color=blue,smooth,ultra thick, domain=-10:10] {x*log2(x)};
\end{axis}
\end{tikzpicture}
\newline

\section{Questão 5}
\lstinputlisting[language=Java]{pesquisa.java}

\lstinputlisting[language=Java]{pesquisaBinaria.java}

\section{Questão 6}
\hspace{10mm} O código verifica se a letra 'c' tem o código ASCII  65, 69, 73, 79, 85, 97, 101, 105, 111 ou 117.

\section{Questão 7}
\hspace{4mm} O primeiro código verifica se a letra de entrada tem o código ASCII  65, 69, 73, 79, 85, 97, 101, 105, 111 ou 117.

Os outros dois códigos transformam o \emph{char} de entrada em minúsculo e verifica se essas letras são vogais.
    
\section{Questão 8}
\hspace{4mm} O primeiro código verifica se o char de entrada é uma letra.

O segundo código verifica se é uma vogal.

\section{Questão 9}
\lstinputlisting[language=Java]{menos.java}

A primeira função subtrai depois de retornar seu valor. Enquanto, na segunda primeiro será feito a subtração, depois será retornado o valor.

\section{Questão 10}
\hspace{4mm} O $b$ a cada soma irá mudar de 0 para 1 e de volta para 0. A variável $s$ irá somar 1 até 32767 e irá para -32768. A variável $i$ irá de -32768 até 32767. E, por fim, $l$ de -2147483648 até 2147483647.

\section{Questão 11}
\hspace{4mm} O uso do operando $>>$ faz multiplicações e divisões por 2. Dessa forma, $23 >> 1$, será 46. Já $23 << 1$ será $\floor{\frac{23}{2}}$, pois divisões que geram resto são ignorados! Deixando o resultado como 11.

\end{document}
