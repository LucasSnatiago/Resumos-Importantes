\documentclass{article}
\usepackage[utf8]{inputenc}
\usepackage{mathtools}
\usepackage{listings}
\usepackage{xcolor}

\definecolor{codegreen}{rgb}{0,0.6,0}
\definecolor{codegray}{rgb}{0.5,0.5,0.5}
\definecolor{codepurple}{rgb}{0.58,0,0.82}
\definecolor{backcolour}{rgb}{0.95,0.95,0.92}

\lstdefinestyle{codigo}{
    backgroundcolor=\color{backcolour},   
    commentstyle=\color{codegreen},
    keywordstyle=\color{magenta},
    numberstyle=\tiny\color{codegray},
    stringstyle=\color{codepurple},
    basicstyle=\ttfamily\footnotesize,
    breakatwhitespace=false,         
    breaklines=true,                 
    captionpos=b,                    
    keepspaces=true,                 
    numbers=left,                    
    numbersep=5pt,                  
    showspaces=false,                
    showstringspaces=false,
    showtabs=false,                  
    tabsize=2
}

\lstset{style=codigo}

\title{Resumo sobre somatórios}
\author{Lucas Santiago de Oliveira}
\date{Março de 2020}

\begin{document}

\maketitle

\section{O que é somatório}
\hspace{4mm} O somatório é um símbolo matemático ($\Sigma$) que designa a soma de um conjunto de termos de uma sequência. 

\begin{figure}[h]
\centering
\includegraphics[width=5cm]{Somatorio.png}
\label{Figura:Exemplo de Somatório}
\end{figure}

Onde \emph{x} é uma sequência que pode ser calculada, \emph{i} é o índice do somatório, \emph{m} é o valor inicial e \emph{n} é o valor final.

Para exemplificar:

\begin{figure}[h]
\centering
\includegraphics[width=5cm]{sequencia.png}
\label{Figura:Exemplo de sequência}
\end{figure}

\section{Algumas aplicações}

\subsection{Cálculo de média aritmética}
\hspace{4mm} Somatório pode ser facilmente usado para cálculo de médias aritméticas:

\begin{figure}[h]
\centering
\includegraphics[width=2cm]{media.png}
\label{Figura:Exemplo de sequência}
\end{figure}

Nesse caso a média é calculada da seguinte forma: i começa em 1, n como número de elementos, \emph{x} sendo os valores separadamente.

Para simplificar, usarei alguns valores: 1, 3, 5, 7, 9. Dessa forma, n será 5 (há 5 elementos). \emph{x} no índice 1 é 1, no índice 2 é 3, no índice 3 é 5 e assim por diante. Será somado todos esses elementos, depois será dividido todos eles por $n$ (número de elementos). Dando assim a média entre todos esses termos:
\newpage
\hspace{2cm}$X = \frac{1}{n} \sum_{i=1}^{5} x_i$ \hspace{2cm} $x_i =$ \{1, 3, 5, 7, 9\}

\vspace{5mm} Resposta final: $5$


\subsection{Uso na computação}
\hspace{4mm} O principal uso do somatório na computação são os \emph{loops for}.
\lstinputlisting[language=C]{for.c}

Seguindo do exemplo de código: o comando \emph{for} repete algum conjunto de instruções pelo número de vezes que lhe for pedido. Usando dos conceitos do somatório: o primeiro conjunto de instrução que precisa ser colocado no \emph{for} é o valor inicial, o próximo é a condição de execução (valor final) e, por fim, qual será o quanto será acrescentado no \emph{i} a cada iteração.

\end{document}