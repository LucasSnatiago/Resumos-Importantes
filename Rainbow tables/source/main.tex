\documentclass[12pt]{article}
\usepackage[utf8]{inputenc}
\usepackage[brazil]{babel}
\usepackage{setspace}
\usepackage{mathtools}
\usepackage{pgfplots}
\usepackage{listings}
\usepackage{xcolor}
\usepackage{natbib}
\usepackage{graphicx}
\usepackage[margin=2cm]{geometry}

\DeclarePairedDelimiter\ceil{\lceil}{\rceil}
\DeclarePairedDelimiter\floor{\lfloor}{\rfloor}

\definecolor{codegreen}{rgb}{0,0.6,0}
\definecolor{codegray}{rgb}{0.5,0.5,0.5}
\definecolor{codepurple}{rgb}{0.58,0,0.82}
\definecolor{backcolour}{rgb}{0.95,0.95,0.92}

\lstdefinestyle{codigo}{
    numberstyle=\tiny,
    basicstyle=\ttfamily\footnotesize,
    breakatwhitespace=false,         
    breaklines=true,                 
    captionpos=b,                    
    keepspaces=true,                 
    numbers=left,                    
    numbersep=5pt,                  
    showspaces=false,                
    showstringspaces=false,
    showtabs=false,                  
    tabsize=2
}

\lstset{style=codigo}

\setstretch{1.5}
\pgfplotsset{width=10cm,compat=1.9}

\pagenumbering{gobble}
\clearpage
\thispagestyle{empty}

\title{Tabelas Arco-Íris - Hash Criptográfico}
\author{Lucas Santiago de Oliveira}
\date{Novembro de 2020}



\begin{document}
\maketitle

\hspace*{4pt} Ataques de força bruta eram muito comuns em bancos de dados para se obter as senhas dos usuários que estavam lá.
Com o tempo, os algoritmos criptográficos foram se tornando mais complexos e esse tipo de ataque foi perdendo popularidade para outros.
Para entender como uma tabela Arco-Íris (\emph{Rainbow Table}) funciona primeiro precisa-se entender como funciona um hash criptográfico moderno
\emph{(Essa discução ficará para outro documento)}.

Tabelas Arco-Íris foram uma saida rápida para quebrar banco de dados, sem necessidade de grande poder computacional ou tempo. Essas tabelas consistem 
em precomputar vários conjutos de texto ou senhas comuns previamente e salvar em um arquivo. Assim, caso você precise quebrar alguma senha será apenas
preciso verificar o hash encontrado no banco com os hashs que estão salvos no arquivo. Para ser mais rápido, há a possibilidade de ordenar a tabela e
fazer uma simples busca binária. Dessa forma, 1 hash pode ser pesquisado em um banco com 1.000.000 de elementos em apenas 20 testes.

Essa forma de armazenamento de senha se mostrou rapidamente ineficaz e algumas soluções apareceram.
Para aumentar mais a segurança é adicionado uma senha gerada aleatoriamente no inicio da senha do usuário antes de colocar a senha sobre alguma função de hash além de passar
a senha por essa função mais de uma vez. O valor da senha gerada aleatoriamente e do número de ciclos do hash podem ser salvos no arquivo sem nenhum problema.
A dificuldade computacional de pegar todas as senhas mais comuns do mundo e adicionar a senha aleatória do banco de dados mais o número de ciclos torna absurdamente caro
que um banco de dados vazado seja comprometido rapidamente.

\end{document}